\section{Constraints}

\subsection{Technical Constraints}

\textbf{TC-1: Git's Content Addressability Model} \\
Git uses SHA-1 hashing for object addressing. While SHA-1 has known collision vulnerabilities, Git is transitioning to SHA-256. The system assumes SHA-1 is ``good enough'' for content addressing (not for security-critical signing).

\textbf{Mitigation:} Use GPG signing (\texttt{CMS\_SIGN=1}) for cryptographic non-repudiation.

\textbf{TC-2: Filesystem I/O Performance} \\
All Git operations are ultimately filesystem operations. Performance is bounded by disk I/O, especially for large repositories.

\textbf{Mitigation:} Content is stored as commit messages (small), not files (large). Asset chunking (256KB) reduces blob size.

\textbf{TC-3: POSIX Shell Dependency} \\
The \texttt{@git-stunts/plumbing} module executes Git via shell commands (\texttt{child\_process.spawn}). This requires a POSIX-compliant shell and Git CLI.

\textbf{Mitigation:} All tests run in Docker (Alpine Linux) to ensure consistent environments.

\textbf{TC-4: No Database Indexes} \\
Traditional databases provide B-tree indexes for fast lookups. Git's ref enumeration is linear (\texttt{O(n)} for listing all refs in a namespace).

\textbf{Mitigation:} Use ref namespaces strategically (e.g., \texttt{refs/\_blog/articles/<slug>}) to avoid polluting the global ref space.

\subsection{Regulatory Constraints}

\textbf{RC-1: GDPR Right to Erasure} \\
Git's immutability conflicts with GDPR's ``right to be forgotten.'' Deleting a commit requires rewriting history, which breaks cryptographic integrity.

\textbf{Mitigation:} Use encrypted assets with key rotation. Deleting the encryption key renders historical content unreadable without altering Git history.

\textbf{RC-2: Cryptographic Export Restrictions} \\
AES-256-GCM encryption may face export restrictions in certain jurisdictions.

\textbf{Mitigation:} The \texttt{@git-stunts/vault} module uses Node's built-in \texttt{crypto} module, which is widely available.

\subsection{Operational Constraints}

\textbf{OC-1: Single-Writer Assumption} \\
Git's ref updates are atomic \textit{locally} but not across distributed clones. Concurrent writes to the same ref can cause conflicts.

\textbf{Mitigation:} Use \textbf{git-stargate} (a companion project) to enforce serialized writes via SSH.

\textbf{OC-2: Repository Growth} \\
Every draft save creates a new commit. Repositories can grow unbounded over time.

\textbf{Mitigation:} Use \texttt{git gc} aggressively. Consider ref pruning for old drafts.